\documentclass[a4 paper]{article}
\usepackage[inner=2.0cm,outer=2.0cm,top=2.5cm,bottom=2.5cm]{geometry}
\usepackage{setspace}
\usepackage[ruled]{algorithm2e}
\usepackage[rgb]{xcolor}
\usepackage{verbatim}
\usepackage{subcaption}
\usepackage{amsgen,amsmath,amstext,amsbsy,amsopn,tikz,amssymb,tkz-linknodes}
\usepackage{fancyhdr}
\usepackage[colorlinks=true, urlcolor=blue,  linkcolor=blue, citecolor=blue]{hyperref}
\usepackage[colorinlistoftodos]{todonotes}
\usepackage{rotating}
\usepackage{booktabs}
\newcommand{\ra}[1]{\renewcommand{\arraystretch}{#1}}

\newtheorem{thm}{Theorem}[section]
\newtheorem{prop}[thm]{Proposition}
\newtheorem{lem}[thm]{Lemma}
\newtheorem{cor}[thm]{Corollary}
\newtheorem{defn}[thm]{Definition}
\newtheorem{rem}[thm]{Remark}
\numberwithin{equation}{section}

\newcommand{\homework}[6]{
   \pagestyle{myheadings}
   \thispagestyle{plain}
   \newpage
   \setcounter{page}{1}
   \noindent
   \begin{center}
   \framebox{
      \vbox{\vspace{2mm}
    \hbox to 6.28in { {\bf CSE 211:~Discrete Mathematics \hfill {\small (#2)}} }
       \vspace{6mm}
       \hbox to 6.28in { {\Large \hfill #1  \hfill} }
       \vspace{6mm}
       \hbox to 6.28in { {\it Instructor: {\rm #3} \hfill Name: Barış Batuhan Bolat {\rm #5} \hfill Student Id: 210104004029 {\rm #6}} \hfill}
       \hbox to 6.28in { {\it Assistant: #4  \hfill #6}}
      \vspace{2mm}}
   }
   \end{center}
   \markboth{#5 -- #1}{#5 -- #1}
   \vspace*{4mm}
}

\newcommand{\problem}[2]{~\\\fbox{\textbf{Problem #1}}\hfill (#2 points)\newline\newline}
\newcommand{\subproblem}[1]{~\newline\textbf{(#1)}}
\newcommand{\D}{\mathcal{D}}
\newcommand{\Hy}{\mathcal{H}}
\newcommand{\VS}{\textrm{VS}}
\newcommand{\solution}{~\newline\textbf{\textit{(Solution)}} }

\newcommand{\bbF}{\mathbb{F}}
\newcommand{\bbX}{\mathbb{X}}
\newcommand{\bI}{\mathbf{I}}
\newcommand{\bX}{\mathbf{X}}
\newcommand{\bY}{\mathbf{Y}}
\newcommand{\bepsilon}{\boldsymbol{\epsilon}}
\newcommand{\balpha}{\boldsymbol{\alpha}}
\newcommand{\bbeta}{\boldsymbol{\beta}}
\newcommand{\0}{\mathbf{0}}


\begin{document}
\homework{Homework \#1}{Due: 30/10/22}{Dr. Zafeirakis Zafeirakopoulos}{Başak Karakaş}{}{}
\textbf{Course Policy}: Read all the instructions below carefully before you start working on the assignment, and before you make a submission.
\begin{itemize}
\item It is not a group homework. Do not share your answers to anyone in any circumstance. Any cheating means at least -100 for both sides. 
\item Do not take any information from Internet.
\item No late homework will be accepted. 
\item For any questions about the homework, send an email to bkarakas2018@gtu.edu.tr
\item Use LaTeX. You can work on the tex file shared with you in the assignment document.
\item Submit both the tex and pdf files into Homework1. Name of the files should be "\emph{SurnameName$\_$Id.tex}" and "\emph{SurnameName$\_$Id.pdf}".
\end{itemize}

\problem{1: Sets}{3+3+3+3+3=15}
Which of the following sets are equal? Show your work step by step.\newline
\subproblem{a} $\{$t : t is a root of $x^2$ – 6x + 8 = 0$\}$
\newline
\subproblem{b} $\{$y : y is a real number in the closed interval [2, 3]$\}$
\newline
\subproblem{c} $\{$4, 2, 5, 4$\}$
\newline
\subproblem{d} $\{$4, 5, 7, 2$\}$ - $\{$5, 7$\}$
\newline
\subproblem{e} $\{$q: q is either the number of sides of a rectangle or the number of digits in any integer between 11 and 99$\}$\\
\solution
\newline

(a) For identifying the set(a)  we need to find of $x^2-6x+8=0$
\newline
\begin{eqnarray}
x^2-6x+8=0\nonumber\\
x^2-4x-2x+8=0\nonumber\\
x(x-4)-2(x-4)=0\nonumber\\
(x-2)(x-4)=0\nonumber\\
x=2 , x=4\nonumber\\
\nonumber\end{eqnarray}

According to steps I wrote above the results of $x^2-6x+8=0$ are $2$ and $4$. \\

And set of (a) = $\{2,4\}$.\\

The sets of b, c and d are certain.\\
\begin{eqnarray}
(b)=[2,3]\nonumber\\
(c)=\{4,2,5,4\} \equiv \{2,4,5\}\nonumber\\
(d)=\{4,5,7,2\} - \{5,7\} = \{2,4\}\nonumber\\
\nonumber\end{eqnarray}

(e) 

\begin{itemize}
    \item Number of sides of a rectangle = 4 \\
    \item Number of digits in any integer between 11 and 99 = 2
\end{itemize}

The set of (e) = {2,4}\\

\textbf{Thus}\\
\begin{center}
(a) $\equiv$ (d) $\equiv$ (e)\hspace{1 cm}a = \{2,4\},\\
\hspace{3.3 cm}d = \{2,4\},\\
\hspace{3.3 cm}e = \{2,4\},
\end{center}




\newpage
\problem{2: Cardinality of Sets}{2+2+2+2=8}
What is the cardinality of each of these sets? Explain your answers.\\
\subproblem{a} $\{\emptyset\}$\\
\subproblem{b} $\{\emptyset,\{\emptyset\}\}$\\
\subproblem{c} $\{\emptyset,\{\emptyset,\{\emptyset\}\}\}$\\
\subproblem{d} $\{\emptyset,\{\emptyset,\{\emptyset,\{\emptyset\}\}\}\}$\\
\solution

(a) Cardinality is 1. This set containes an empty set($\emptyset$).\\

(b) Cardinality is 2. This set containes an empty set($\emptyset$) and another set that contains empty set($\{\emptyset\}$).\\

(c) Cardinality is 2. This set containes an empty set($\emptyset$) and ($\{\emptyset,\{\emptyset\}\}$).$\{\emptyset,\{\emptyset\}\}$ consider as 1.\\

(d) Cardinality is 2. This set containes an empty set($\emptyset$) and ($\{\emptyset,\{\emptyset,\{\emptyset\}\}\}$). $\{\emptyset,\{\emptyset,\{\emptyset\}\}\}$ consider as 1.
\newline
\newline
\newline
\newline
\newline
\newline
\newline
\newline
\newline




\problem{3: Cartesian Product of Sets}{15}
Explain why (A $\times$ B) $\times$ (C $\times$ D) and A $\times$ (B $\times$ C) $\times$ D are not the same.\\
\solution
\newline

Let $A\times B = K$ and $C\times B = L$\\
\begin{eqnarray}
K\times L = \{(k,l); k\in K , l\in L\nonumber\}\\
k = (a,b), l = (c,d)\nonumber\\
(A\times B)\times (C\times D) = \{((a,b),(c,d));a\in A,b\in B,c\in C,d\in D\}\nonumber\\
\nonumber\end{eqnarray}

Let $B\times C = M$\\
\begin{eqnarray}
M = \{m;m\in M\}\nonumber\\
A\times M\times D = \{(a,m,d);a\in A,m\in M,d\in D\}\nonumber\\
m = (b,c)\nonumber\\
A\times (B\times C)\times D = \{(a,(b,c),d);a\in A,b\in B,c\in C,d\in D\}\nonumber\\
\nonumber\end{eqnarray}
\begin{center}
    $((a,b),(c,d)) \neq (a,(b,c),d)$\\
    \textbf{Result}\\
    $(A\times B)\times (C\times D) \neq A\times (B\times C)\times D$\\
\end{center}

\newpage
\problem{4: Cartesian Product of Sets in Algorithms }{25}
Let A, B and C be sets which have different cardinalities. Let (p, q, r) be each triple of A $\times$ B $\times$ C where p $\in$ A, q $\in$ B and r $\in$ C. Design an algorithm which finds all the triples that are satisfying the criteria: p $\leq$ q and q $\geq$ r. Write the pseudo code of the algorithm in your solution.\newline
\newline
For example: Let the set A, B and C be as A = $\{$ 3, 5, 7 $\}$, B = $\{$ 3, 6 $\}$ and C = $\{$ 4, 6, 9 $\}$. Then the output should be : $\{$ (3, 6, 4), (3, 6, 6), (5, 6, 4), (5, 6, 6) $\}$. \newline
\newline
(Note: Assume that you have sets of A, B, C as an input argument.)\newline
\solution

\begin{algorithm}
\SetAlgoLined
\KwIn{The sets of A, B, C}
\eIf{write a condition}{
    Statements
}{
 Statements
}
 When you want to write a for loop, you can use: \newline
\For{write a condition}{

}
 When you want to write a while loop, you can use: \newline
\While{write a condition}{
If you need to return, use \Return
}
 For any additional things you have to do while writing your pseudo code, Google "How to use algorithm2e in Latex?".
\caption{Pseudo Code of Your Algorithm}
\end{algorithm}

\begin{algorithm}
\SetAlgoLined
\KwIn{The sets of A, B, C}
\For {p in A}{
    \For {q in B}{
         \For {r in C}{
            \eIf{$p \leq q$ and $q\geq r$}{
                print $(p,q,r),$
            }{
             continue
            }
         }
    }
}
\caption{Solution}
\end{algorithm}

\newpage
\problem{5: Functions}{16}
If f and f $\circ$ g are one-to-one, does it follow that g is one-to-one? Justify your answer.\\
\solution\\
\begin{itemize}
    \item Assume g is not one-to-one then $a \neq b$ in such that $g(a) = g(b)$
    \item f and $f\circ g$ are one-to-one $f\circ g(a) = f\circ g(b)$ then $g(a) = g(b)$
    \item With $a\neq b$,which contradicts $f\circ g$ one-to-one
    \item Then $g$ must be one-to-one
\end{itemize}

\problem{6: Functions}{7+7+7=21}
Determine whether the function $f:$ $\mathbb{Z}\times\mathbb{Z}\to\mathbb{Z}$ is onto if\\
\subproblem{a} $f(m,n)=2m-n$\\
\subproblem{b} $f(m,n)=m^2-n^2$\\
\subproblem{c} $f(m,n)=\mid m\mid - \mid n\mid$\\
\solution\\

The function $f:A\xrightarrow[]{} B$ is onto if range of function same as domain. $\Rightarrow f(A) = B$\\

(a) For any pair $(0,n)\in \mathbb{Z}\times \mathbb{Z}$
\begin{eqnarray}
f(0,n) = 2(0)-n\nonumber\\
f(0,n) = n\nonumber\\
Therefore ,\forall n\in \mathbb{Z}
\nonumber\end{eqnarray}

So $f$ is \textbf{onto}.\\

(b) $f(m,n) = m^2-n^2$ is a difference of two perfect squares.
\begin{eqnarray}
f(1,0) = 1^2-0^2 = 1\nonumber\\
f(2,1) = 2^2-1^2 = 3\nonumber\\
f(3,2) = 3^2-2^2 = 5\nonumber\\
&\vdots\nonumber\\
\nonumber\end{eqnarray}


According to the above, some elements in the image set do not have a pre image in the domain.(For ex. 2) So $f$ is \textbf{not onto}.\\

(c) For any pair $(m,0) = \mathbb{Z}\times \mathbb{Z}$
\begin{align}
f(m,0) &= \mid m\mid - \mid 0\mid\nonumber\\
       &= \mid m\mid\nonumber\\
       &= \pm m\nonumber\\
Therefore ,\forall \pm m\in \mathbb{Z}\nonumber\\
\end{align}

So f is \textbf{onto}

\newpage
\problem{7: Functions}{Bonus 20}
Suppose that $f$ is a function from $A$ to $B$, where $A$ and $B$ are finite sets with $\mid A\mid=\mid B\mid$. Show that $f$ is one-to-one if and only if it is onto.\\
\solution
\newline
\end{document}